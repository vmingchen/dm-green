\section{Related Work} 
\label{sec:related} 

Our study provides a multi-disk (hybrid) system in the purpose of
saving energy. To achieve that, we use energy-efficient disk as cache
and perform data migration and grouping using information obtained
from traces study. We discuss related work that covers the following
aspects: save energy, hybrid disk, data grouping and workload trace
study.

\textbf{Save Energy}. Energy efficiency of storage systems is an
interesting problem that is being actively researched.
\cite{fast10fsgreen} studies the power consumption of servers when
different workloads are running.  \cite{systor09greencomp} presents
empirical analysis of how lossless compression influences power
consumption of servers. \cite{fast12emc-energy} presents the first
study of power consumption of really running enterprise-scale backup
systems. A category of method to save energy is using data replication
to decrease either disk head movement \cite{huangfs2,
pred_data_grouping} or the number of active disks being used
\cite{Weddle07paraid}. 

\textbf{Hybrid disk}. \cite{Bisson07ahybrid, Debnath_SkimpyStash,
Debnath_Bloomflash, flashcache_experiments} are studying hybrid disk
models, where SSD is used as cache. Among them, only
\cite{Bisson07ahybrid} is focusing on the energy efficiency aspect.
Although \cite{Zhu04reducingenergy} is also studying power aware
cache, their cache is in volatile RAM instead of non-volatile disk.
\cite{slow_fast} is also a hybrid storage system that combines low and
high speed disks to save energy.

\textbf{Data grouping}. \cite{Wildani_grouping} presents data
grouping of working set, however, it only contains analysis of offline
traces. It is not implemented or experimented on real system. The
Energy-Efficient File System (EEFS) groups files with high temporal
access locality \cite{Li06highperformance}. \cite{pred_data_grouping}
also performed predictive data grouping to reduce head movements for
saving energy. \cite{migration04} used the fact that regularly only a
small subset of data is accessed by a system, and migrated frequently
accessed data to a small number of active disks, keeping the remaining
disks off, which is a kind of data grouping.


\textbf{Trace study}. Trace extraction is a popular method used by
system researchers to analyze the system behavior. The analysis can be
used to optimize several important features of the system like CPU and
memory utilization, I/O throughput, latency, etc. Several reseachers
have used trace analysis to find correlations between disk block
accesses. Z. Li et al \cite{Li04c-miner} used data mining techniques
to find block correlations. T. Li et al \cite{Li_model} used block
traces to to run-time modeling to estimate the power consumption of
servers. Douglas et al. \cite{Douglis_95} used efficient algorithms
based on machine learning techniques on block traces to spin down
disks and extend battery life in mobile computers. V. Torasov
\cite{fast12t2m} used trace analysis to evaluate performance and
energy efficiency in file system server workloads extensions.


%%%%%%%%%%%%%%%%%%%%%%%%%%%%%%%%%%%%%%%%%%%%%%%%%%%%%%%%%%%%%%%%%%%%%%%%%%%%%%
%% For Emacs:
% Local variables:
% fill-column: 70
% End:
%%%%%%%%%%%%%%%%%%%%%%%%%%%%%%%%%%%%%%%%%%%%%%%%%%%%%%%%%%%%%%%%%%%%%%%%%%%%%%
%% For Vim:
% vim:textwidth=70
%%%%%%%%%%%%%%%%%%%%%%%%%%%%%%%%%%%%%%%%%%%%%%%%%%%%%%%%%%%%%%%%%%%%%%%%%%%%%%
% LocalWords:  SMR HDDs drive's SMRs
