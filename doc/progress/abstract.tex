\begin{abstract}

% dummy entry to force emacs not to indent abstract text
\vspace{0mm}

\todo[color=blue!40,inline]{Zhichao: we have updated the project plan
for the class because of the limited time. Therefore, please make sure 
the abstract matches what will be presented in the report. }

Storage subsystems in servers are slower than other subsystems despite
the fact that they contribute to a large portion of the total power
consumption. Hybrid disks, which integrate different characteristics
of various kinds of disks (e.g., SATA, SAS, SSD) in speed, capacity,
price, and power consumption, is a promising solution to the problem.
While researches on hybrid disk have reported significant performance
boost over traditional magnetic disks, their energy usage is less
studied.  In this project, we consider energy efficiency of storage
systems as important as performance and capacity.  Our goal is to
provide a multi-disk system that consists of different disks, provides
a capacity equal to the sum of all disks, operates at a speed near the
fastest disk, but consume less energy than the sum of all disks when
they operate individually.

To achieve energy efficiency as well as good performance, our
multi-disk system tries to map more frequently accessed blocks (i.e.,
hot data) to energy-efficient, fast, but small disks such as SSD.
Conversely, cold data goes to inefficient, slow, but large disks such
as SATA.  This exploits spatial locality of data and has similar
effect in hybrid model where SSD is used as cache disk.  Meanwhile,
temporal locality of data is also be exploited to group blocks that
are likely to be accessed simultaneously.  Instead of scattering
blocks among several disks, we try to map them to a single disk.  The
combination of spatial and temporal locality allows I/O be served
using fewer disks and other disks can spin down to save energy.  To
identify hot data and block groups, workload-specific traces are
analyzed offline. Online statistics will also be gathered and used by
heuristic algorithm for adaption to hot data and block groups as they
evolve over time. We are also exploring naive direct IO map without
any knowledge of the workloads as baseline approach. 

\end{abstract}

%%%%%%%%%%%%%%%%%%%%%%%%%%%%%%%%%%%%%%%%%%%%%%%%%%%%%%%%%%%%%%%%%%%%%%%%%%%%%%
%% For Emacs:
% Local variables:
% fill-column: 70
% End:
%%%%%%%%%%%%%%%%%%%%%%%%%%%%%%%%%%%%%%%%%%%%%%%%%%%%%%%%%%%%%%%%%%%%%%%%%%%%%%
%% For Vim:
% vim:textwidth=70
%%%%%%%%%%%%%%%%%%%%%%%%%%%%%%%%%%%%%%%%%%%%%%%%%%%%%%%%%%%%%%%%%%%%%%%%%%%%%%
% LocalWords:  
